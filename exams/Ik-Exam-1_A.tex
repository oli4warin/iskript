\documentclass[11pt]{exam}
 % Exam Template for UMTYMP and Math Department courses
%
% Using Philip Hirschhorn's exam.cls: http://www-math.mit.edu/~psh/#ExamCls
%
% run pdflatex on a finished exam at least three times to do the grading table on front page.
%
%%%%%%%%%%%%%%%%%%%%%%%%%%%%%%%%%%%%%%%%%%%%%%%%%%%%%%%%%%%%%%%%%%%%%%%%%%%%%%%%%%%%%%%%%%%%%%

% These lines can probably stay unchanged, although you can remove the last
% two packages if you're not making pictures with tikz.

\usepackage{hyperref}
\RequirePackage{amssymb, amsfonts, amsmath, latexsym, verbatim, xspace, setspace}
\RequirePackage{tikz, pgflibraryplotmarks}
\usepackage{multicol,fancybox}


%By default, the created grids are in black. However, if you give the commands

\usepackage{color} 
\colorgrids

%then the grids will be in color, by default a light gray. That default color was defined by the command

\definecolor{GridColor}{gray}{0.2}

% By default LaTeX uses large margins.  This doesn't work well on exams; problems
% end up in the "middle" of the page, reducing the amount of space for students
% to work on them.
\usepackage[margin=1in]{geometry}

\usepackage{listings}
\lstset{language=Python}


\definecolor{codegreen}{rgb}{0,0.6,0}
\definecolor{codegray}{rgb}{0.9,0.9,0.9}
\definecolor{codepurple}{rgb}{0.58,0,0.82}
\definecolor{backcolour}{rgb}{0.95,0.95,0.92}

\lstdefinestyle{mystyle}{
%	morekeywords={forward,turn},
    backgroundcolor=\color{codegray},
    commentstyle=\color{codegreen},
%    keywordstyle=\color{codegreen},
    numberstyle=\tiny\color{gray},
    stringstyle=\color{codepurple},
    basicstyle=\footnotesize,
    identifierstyle=\color{blue},
    stringstyle=\color{orange},
    breakatwhitespace=false,
    breaklines=true,
    captionpos=b,
    keepspaces=true,
    numbers=left,
    numbersep=5pt,
    showspaces=false,
    showstringspaces=false,
    showtabs=false,
    tabsize=2
}

\lstset{style=mystyle}

% Here's where you edit the Class, Exam, Date, etc.
\newcommand{\class}{Class: 1E}
\newcommand{\term}{Fall 2021}
\newcommand{\examnum}{Exam 1}
\newcommand{\examdate}{23.9.2021}
\newcommand{\timelimit}{90 Minutes}
\newcommand{\aids}{Head}


% For an exam, single spacing is most appropriate
\singlespacing
% \onehalfspacing
% \doublespacing

% For an exam, we generally want to turn off paragraph indentation
\parindent 0ex
\setlength{\fboxsep}{15pt}
\setlength{\gridsize}{4mm}



\newcommand{\gridandpoints}{\fillwithgrid{\fill}
\hfill Points: \framebox[1.5cm]{} out of \pointsofquestion{\thequestion}
\newpage}



\newcommand{\centfig}[2]{\begin{center}
\includegraphics[width=#1\textwidth]{#2}
\end{center}}



\usepackage{wasysym}












\newcommand{\dg}{^{\circ}}



\newcommand{\vecdd}[2]
{\begin{pmatrix}
#1 \\
#2
\end{pmatrix}}


\newcommand{\vecddd}[3]{\begin{pmatrix}
    {#1}\\
    {#2}\\
    {#3}
\end{pmatrix}
}



\newcommand{\qtext}[1]{\quad\text{#1}\quad}






\begin{document} 

% These commands set up the running header on the top of the exam pages
\pagestyle{head}
\firstpageheader{}{}{}
\runningheader{\class}{\examnum\ - Page \thepage\ of \numpages}{\examdate}
\runningheadrule

\begin{flushright}
\begin{tabular}{p{2.8in} r l}
\textbf{\class} & \textbf{Name: } \hrulefill  & \\
\textbf{\term} &&\\
\textbf{\examnum} &&\\
\textbf{\examdate} &&\\
\textbf{Time Limit: \timelimit} & \textbf{Aids: \aids}  \hspace*{6cm} & 
\end{tabular}\\
\end{flushright}
\rule[1ex]{\textwidth}{.1pt}


This exam contains \numpages\ pages (including this cover page) and
\numquestions\ problems.  Check to see if any pages are missing.  Enter
all requested information on the top of this page.\\

You are required to show your work on each problem on this exam.  The following rules apply:\\

\begin{minipage}[t]{3.7in}
\vspace{0pt}
\begin{itemize}

\item \textbf{Organize your work}, in a reasonably neat and coherent way, in
the space provided. Work scattered all over the page without a clear ordering will receive very little credit.  

%\item \textbf{Mysterious or unsupported answers will not receive full
%credit}.  A correct answer, unsupported by calculations, explanation,
%or algebraic work will receive no credit; an incorrect answer supported
%by substantially correct calculations and explanations might still receive
%partial credit.


%\item {\bf Show your use of the calculator.}


\end{itemize}

\end{minipage}
\hfill
%\begin{minipage}[t]{2.3in}
%\vspace{0pt}
%\cellwidth{3em}
\gradetablestretch{2}
\vqword{Problem}
\addpoints % required here by exam.cls, even though questions haven't started yet.	

\begin{center}
\gradetable[h] %[pages]  % Use [pages] to have grading table by page instead of question
\end{center}














%\end{minipage}







\vfill

\[
\fbox{$
\text{Final Grade} = \dfrac{\text{Points Made}}{\numpoints}\cdot 5+1 \text{ (rounded to one decimal place)}=\doublebox{\mbox{\hspace*{1cm}}}
$}
\]
\vfill















\newpage % End of cover page

%%%%%%%%%%%%%%%%%%%%%%%%%%%%%%%%%%%%%%%%%%%%%%%%%%%%%%%%%%%%%%%%%%%%%%%%%%%%%%%%%%%%%
%
% See http://www-math.mit.edu/~psh/#ExamCls for full documentation, but the questions
% below give an idea of how to write questions [with parts] and have the points
% tracked automatically on the cover page.
%
%
%%%%%%%%%%%%%%%%%%%%%%%%%%%%%%%%%%%%%%%%%%%%%%%%%%%%%%%%%%%%%%%%%%%%%%%%%%%%%%%%%%%%%









\newcommand{\iu}{{\bf i}}

\begin{questions}


% Basic question
\addpoints




\question[5] For the following questions draw the images that the turtle creates. It is not necessary to measure lengths and angles precisely -- a rough estimate  is fine. You also do not have to consider the boundaries of the graphing area from the jupyter notebooks.

\begin{parts}
\part What does the following code draw? 
\begin{lstlisting}
from gymmu.turtle import *

make_turtle()

forward(100)
turn(90)
forward(-100)
turn(-90)
forward(100)
turn(-90)
forward(100)
turn(90)
forward(100)

show()
\end{lstlisting}
\part What does the following code draw? 
\begin{lstlisting}
from gymmu.turtle import *

make_turtle()

for i in range(8):
	forward(100)
	turn(360/8)

show()
\end{lstlisting}

\part What does the following code draw? 
\begin{lstlisting}
from gymmu.turtle import *

def draw_hat():
    turn(60)
    forward(100)
    turn(-120)
    forward(100)
    turn(60)

make_turtle()

for i in range(4):
    draw_hat()
    forward(50)

show()
\end{lstlisting}
\end{parts} 

\gridandpoints


\question[5] Write code that draws the following images. Note: full credit is given for  code that efficiently uses the concepts learned.

\begin{parts}
\part 
\part 
\end{parts} 


\gridandpoints


\question[2] Find the error in the shown code.


\gridandpoints



\question[4]  







\end{questions}
\end{document}
