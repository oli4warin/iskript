\documentclass[11pt,a4paper]{report}

\usepackage{xcolor}
\def\farbe{cyan}

\usepackage{dclecture}

\usepackage{listings}
\lstset{language=html}


%\definecolor{codegreen}{rgb}{0,0.6,0}
%\definecolor{codegray}{rgb}{0.9,0.9,0.9}
%\definecolor{codepurple}{rgb}{0.58,0,0.82}
%\definecolor{backcolour}{rgb}{0.95,0.95,0.92}
%
%\lstdefinestyle{mystyle}{
%%	morekeywords={forward,turn},
%    backgroundcolor=\color{codegray},
%    commentstyle=\color{codegreen},
%%    keywordstyle=\color{codegreen},
%    numberstyle=\tiny\color{gray},
%    stringstyle=\color{codepurple},
%    basicstyle=\footnotesize,
%    identifierstyle=\color{blue},
%    stringstyle=\color{orange},
%    breakatwhitespace=false,
%    breaklines=true,
%    captionpos=b,
%    keepspaces=true,
%    numbers=left,
%    numbersep=5pt,
%    showspaces=false,
%    showstringspaces=false,
%    showtabs=false,
%    tabsize=2
%}
%
%\lstset{style=mystyle}




%%% Fancy Header and Footer
\renewcommand{\headrule}{\vbox to 0pt{\hbox to\headwidth{\color{\farbe}\rule{\headwidth}{1pt}}\vss}}
\pagestyle{fancy} %eigener Seitenstil
\fancyhf{} %alle Kopf- und Fusszeilenfelder bereinigen
\fancyhead[C]{Computer Science} %Kopfzeile mitte
%\fancyhead[R]{\includegraphics[width=0.2cm]{x.png}}
\fancyfoot[C]{\thepage}







\begin{document}
\section{HTML/CSS}




\subsection{The Chat}

\begin{ex}
Download and run our code editor Visual Studio Code (VSCode for those on the inside). You can find the app with the link below. \\
\url{https://code.visualstudio.com/download}

Since VSCode now belongs to Microsoft there is also a "clean" (i.e. does not send your data to Redmond) version that is slightly trickier to install called VSCodium. \\
\url{https://github.com/VSCodium/vscodium/releases}
\end{ex}

\begin{ex}
\begin{enumerate}
\item Download the chat folder "chat" via the zip-file in OneNote. Place in your CS folder and double click to extract.
\item Open the html-file with your web browser (double clicking usually does this by default).
\item Register with a username and a password {\bf Attention: This is not a secure site. Do not use a real password.}
\item Spend a few minutes to exchange some messages with your colleagues. You will have to exchange usernames to  send them a message.
%\item The built in encryption is a Vigenère code. You can  select a keyword and send encrypted messages.
\end{enumerate}

\end{ex}


\begin{ex}
Take a piece of paper (or your favorite sketching tool) and start designing your chat. You can then edit the html code and the css file to change the contents and styling of your chat. 

\end{ex}

\begin{ex}
In the CSS file you can see an \texttt{encryption-group}.  This allows you to encrypt and decrypt messages using a Vigenère code. Change the display setting to see the encryption option.
\end{ex}




\end{document}
