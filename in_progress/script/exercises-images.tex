\documentclass[11pt,a4paper]{report}

\usepackage{xcolor}
\def\farbe{cyan}

\usepackage{dclecture}

\usepackage{listings}
\lstset{language=Python}


\definecolor{codegreen}{rgb}{0,0.6,0}
\definecolor{codegray}{rgb}{0.9,0.9,0.9}
\definecolor{codepurple}{rgb}{0.58,0,0.82}
\definecolor{backcolour}{rgb}{0.95,0.95,0.92}

\lstdefinestyle{mystyle}{
%	morekeywords={forward,turn},
    backgroundcolor=\color{codegray},
    commentstyle=\color{codegreen},
%    keywordstyle=\color{codegreen},
    numberstyle=\tiny\color{gray},
    stringstyle=\color{codepurple},
    basicstyle=\footnotesize,
    identifierstyle=\color{blue},
    stringstyle=\color{orange},
    breakatwhitespace=false,
    breaklines=true,
    captionpos=b,
    keepspaces=true,
    numbers=left,
    numbersep=5pt,
    showspaces=false,
    showstringspaces=false,
    showtabs=false,
    tabsize=2
}

\lstset{style=mystyle}




%%% Fancy Header and Footer
\renewcommand{\headrule}{\vbox to 0pt{\hbox to\headwidth{\color{\farbe}\rule{\headwidth}{1pt}}\vss}}
\pagestyle{fancy} %eigener Seitenstil
\fancyhf{} %alle Kopf- und Fusszeilenfelder bereinigen
\fancyhead[C]{Computer Science} %Kopfzeile mitte
%\fancyhead[R]{\includegraphics[width=0.2cm]{x.png}}
\fancyfoot[C]{\thepage}







\begin{document}
\section{Images}

\subsection{Basics}

\begin{ex}
Use the example code below as a starting point. Change the code in such a way
that the resulting chessboard pattern has 4 fields per row.

\begin{lstlisting}
from gymmu.images import *

data = [1, 0, 1, 0, 1, 0, 1, 0, 1]
write_image_from_data(data)
\end{lstlisting}
\end{ex}

\begin{ex}
What image does the following code produce? Think about it first, before you
execute the code.

\begin{lstlisting}
from gymmu.images import *

data = []
for i in range(100):
    data.append(i * 0.01)
write_image_from_data(data)
\end{lstlisting}
\end{ex}

\begin{ex}
What does the following code produce? Make an educated guess before you run
your code.

\textbf{Hint:} \% is the modulo operator.


\begin{lstlisting}
from gymmu.images import *

data = []
for i in range(100):
    data.append(i % 2)
write_image_from_data(data)
\end{lstlisting}
\end{ex}


\newpage
\begin{ex}
Can you guess what image this code produces?

\textbf{Hint:} Write down the values for $j$ and $i$.

\begin{lstlisting}
from gymmu.images import *

data = []
for i in range(10):
    for j in range(10):
        data.append((i + j) % 2)
write_image_from_data(data)
\end{lstlisting}
\end{ex}

\begin{ex}
Write code that produces a chessboard pattern with 16 fields per row, where the
top left field is white.
\end{ex}

\begin{ex}
What image does the following code produce?
\begin{lstlisting}
from gymmu.images import *

data = [
    1, 0, 0,
    0, 1, 0,
    0, 0, 1,
    0, 0, 0,
]
write_image_from_data_colored(data)
\end{lstlisting}
\end{ex}



\begin{ex}
Write code that produces an image with the colors \emph{cyan}, \emph{magenta},
\emph{yellow} and \emph{white}.
\end{ex}

\begin{ex}
Using our standard gymmu picture:
\begin{enumerate}
\item Delete the green channel only from the top half of the picture.
\item Delete the green channel from the top half and the red channel from the bottom half.
\item (*hard*) Delete the green channel from the left hand side of the picture.
\item (*even harder*) Make a checkerboard pattern by deleting the green channel from certain squares in the picture.
\end{enumerate} 
\end{ex}

\newpage

\begin{ex}
Describe the image that the following code will produce. Check if your
description was accurate by executing the code.
\begin{lstlisting}
from gymmu.images import *

data = []
for i in range(256):
    data.append(i)
    data.append(0)
    data.append(0)
write_image_from_data_rgb(data)
\end{lstlisting}
\end{ex}



\begin{ex}
Write code that produces an image where the top left pixel is white and the
bottom right pixel is magenta. All the pixels in between follow a magenta
gradient.
\end{ex}
\end{document}
