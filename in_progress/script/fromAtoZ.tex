\documentclass[11pt,a4paper]{report}

\usepackage{xcolor}
\def\farbe{blue}

\usepackage{dclecture}

\usepackage{listings}
\lstset{language=Python}


\definecolor{codegreen}{rgb}{0,0.6,0}
\definecolor{codegray}{rgb}{0.9,0.9,0.9}
\definecolor{codepurple}{rgb}{0.58,0,0.82}
\definecolor{backcolour}{rgb}{0.95,0.95,0.92}

\lstdefinestyle{mystyle}{
%	morekeywords={forward,turn},
    backgroundcolor=\color{codegray},
    commentstyle=\color{codegreen},
%    keywordstyle=\color{codegreen},
    numberstyle=\tiny\color{gray},
    stringstyle=\color{codepurple},
    basicstyle=\footnotesize,
    identifierstyle=\color{blue},
    stringstyle=\color{orange},
    breakatwhitespace=false,
    breaklines=true,
    captionpos=b,
    keepspaces=true,
    numbers=left,
    numbersep=5pt,
    showspaces=false,
    showstringspaces=false,
    showtabs=false,
    tabsize=2
}

\lstset{style=mystyle}




%%% Fancy Header and Footer
\renewcommand{\headrule}{\vbox to 0pt{\hbox to\headwidth{\color{\farbe}\rule{\headwidth}{1pt}}\vss}}
\pagestyle{fancy} %eigener Seitenstil
\fancyhf{} %alle Kopf- und Fusszeilenfelder bereinigen
\fancyhead[C]{Computer Science} %Kopfzeile mitte
%\fancyhead[R]{\includegraphics[width=0.2cm]{x.png}}
\fancyfoot[C]{\thepage}


\newcommand{\bfb}[1]{{\bf \color{blue} #1}}




\begin{document}

\section*{From 1's and 0's to TikTok (or whichever App is trending today)}
\setcounter{section}{1}
A brief overview on how we can manipulate and work with very simple building blocks to create a vast global network. 
\subsection{The 1's and the 0's}
Although some early computers were decimal machines, modern  computers are all binary. The reason for this is mainly engineering: It is relatively easy to make a switch with two positions \emph{on} and \emph{off}. Newer computers still apply this basic idea for their calculations -- just with electricity: low voltage is off (or zero) and high voltage is on (or one).

A basic storage unit of a computer is a \bfb{bit}, which is short for \bfb{binary digit}. Eight bits are called a \bfb{byte}. $1024$ bytes are a \bfb{kilobyte}, because we are dealing with powers of $2$ and $2^{10}=1024$ is closest to $1000$. The same reasoning is applied for megabytes, gigabytes, terabytes etc. 

Since one bit does not encode much information, modern computers use $64$ bits as a basic unit of storage. That means that each number, character and special symbol uses up $64$ bits of memory. This also means that a computer can not calculate with numbers that are too big -- this leads to \bfb{overflow}.

\subsubsection{Exercises}

\begin{ex}
Convert the following binary numbers to decimal.
\begin{multicols}{4}
\begin{enumerate}
\item $10101$
\item $1110$
\item $1000001$
\item $1100110011$
\end{enumerate}
\end{multicols} 
\end{ex}
\sol{
\begin{multicols}{4}
\begin{enumerate}
\item $21$
\item $14$
\item $65$
\item $819$
\end{enumerate}
\end{multicols}
}

\begin{ex}
Convert the following decimal numbers to binary.
\begin{multicols}{4}
\begin{enumerate}
\item $42$
\item $122$
\item $2022$
\item $15$
\end{enumerate}
\end{multicols} 
\end{ex}
\sol{
\begin{multicols}{4}
\begin{enumerate}
\item $101010$
\item $1111010$
\item $11111100110$
\item $1111$
\end{enumerate}
\end{multicols}
}

\begin{ex}
How many bits are in a megabyte?
\end{ex}
\sol{$8'388'608$}

\subsection{Different Bases}
The \bfb{base} of a number system indicates which powers are represented by each position. In our standard \bfb{decimal} system each position is a power of $10$, where the first position are the $1$'s ($10^0=1)$.  In the \bfb{binary} system each position is a power of $2$; so $1$, $2$, $4$, etc.

In this way any natural number can be the base of a number system. We have already seen the \bfb{hexadecimal} system with a base of $16$, but any other number could be the base of a number system. Sometimes you may see a small subscript number indicating the base:
\[
11100_{2} = 28_{10}
\]
\[
122_{3}=17_{10}
\]
\[
122_{7} = 65_{10}
\]



\subsubsection{Exercises}
\begin{ex}
Convert the following numbers to decimal.
\begin{multicols}{4}
\begin{enumerate}
\item $111_{3}$
\item $111_{7}$
\item $1010_{4}$
\item $5_{42}$
\end{enumerate}
\end{multicols} 
\end{ex}
\sol{
\begin{multicols}{4}
\begin{enumerate}
\item $13$
\item $57$
\item $68$
\item $5$
\end{enumerate}
\end{multicols}
}

\subsection{How to Deal with Negative Numbers}
Addition with binary numbers is relatively straightforward: We just add and carry each position as with our decimal addition. For subtraction we use a simple trick: Instead of $7-2$ we write $7+(-2)$. This way if we have a technique for writing $-2$ in binary, then we can also calculate the difference. 

The technique is called \bfb{2's complement} and works as follows: 
\begin{enumerate}
\item[1.] Find the positive binary value for the negative number you want to represent. Write out any leading zeros.
\item[2.] Invert or find the complement of each bit in the number.
\item[3.] Add 1 to this number. 
\end{enumerate}

For this to work each number has to have a fixed

{\bf Example.}  In a computer system with a four bit storage system
\[
1 = 0001_2 \qaq -1 = 1111_2
\]
\[
4 = 0100 \qaq -4 = 1100_2
\]



\subsubsection{Exercises}

\begin{ex}
Use 2's complement to calculate $7-5$ in binary.
\end{ex}

\begin{ex}
Use 2's complement to calculate $5-5$ in binary.
\end{ex}



\subsection{How to Deal with Fractions}
In computing any numbers that are not \bfb{integers} are called \bfb{real}. Since there are only a finite number of storage spaces on a computer there is also only finite precision. The way a computer represents a fraction is by placing a point somewhere in the number. Now the digits to the left signify integers and digits to the right are fractional powers of $2$:
\[
11.11_2 = 3.75
\]


\subsubsection{Exercises}
\begin{ex}
Convert $0.5$ into binary.
\end{ex}
\sol{$0.1_2$}


\begin{ex}
Convert $\dfrac{1}{64}$ into binary.
\end{ex}
\sol{$0.000001_2$}


\begin{ex}
Convert $\dfrac{1}{3}$ into binary.
\end{ex}
\sol{$0.\overline{01}_2$}


\begin{ex}
Convert $0.2$ into binary.
\end{ex}
\sol{$0.\overline{0011}_2$}


\subsubsection{Solutions}
\printcursols

\end{document}
